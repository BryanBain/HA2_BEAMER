\documentclass[t]{beamer}
\usetheme{Copenhagen}
\setbeamertemplate{headline}{} % remove toc from headers
\beamertemplatenavigationsymbolsempty

\usepackage{amsmath, bm, xcolor, tcolorbox}

\title{Complex Fractions}
\author{}
\date{}

\AtBeginSection[]
{
  \begin{frame}
    \frametitle{Objectives}
    \tableofcontents[currentsection]
  \end{frame}
}

\begin{document}

\begin{frame} 
\maketitle
\end{frame}

\section{Simplify Complex Fractions}

\begin{frame}{Complex Fractions}
\begin{tcolorbox}[colframe=red!35!blue, colback=white!30!green, title=\textbf{Complex Fraction.}]
A \textbf{complex fraction} is a rational expression which contains other rational expressions in the numerator and/or denominator.
\end{tcolorbox}
\vspace{12pt}	\pause
Some examples of complex fractions are given below:
\[
\frac{\frac{2}{x} - 3}{\frac{5}{x} + \frac{7}{x}}
\qquad \text{and} \qquad
\frac{\frac{x}{x+1} + \frac{7}{x}}{\frac{3}{2x} + \frac{8}{x-4}}
\]
\end{frame}


\begin{frame}{Simplifying Complex Fractions}
Simplifying a complex fraction involves working with the expression until there is \emph{at most} only one fraction bar in the entire expression.   \newline\\	\pause 

Recall that when we add or subtract fractions with unlike denominators, we need to find a common denominator first.    \newline\\	\pause

Rather than take this approach with complex fractions (which we could, by the way), we are going to clear out our ``tiny" fractions by multiplying every term by the {\color{blue}\textbf{least common tiny denominator}}, or {\color{blue}\textbf{LCTD}}.
\end{frame}

\begin{frame}{Least Common Tiny Denominator}
We find the least common tiny denominator by finding the least common denominator of all of the ``tiny" fractions in the expression. We can then simplify, if possible.  	\newline\\	\pause

\begin{itemize}
\item To find the LCTD of numbers, find the least common multiple of those numbers.   \newline\\ \pause
\item To find the LCTD of variable terms, select the highest power of each term.
\end{itemize}
\end{frame}

\begin{frame}{Example 1}
Simplify each of the following as much as possible.	\newline\\	
(a) \quad $\dfrac{\left(3+\dfrac{1}{x}\right)}{\left(\dfrac{2}{x}+4\right)}$
\onslide<2->{\quad LCTD is $x$}
\begin{align*}
\onslide<3->{\dfrac{\left(3+\dfrac{1}{x}\right)}{\left(\dfrac{2}{x}+4\right)} &\left(\dfrac{x}{x}\right)}
\onslide<4->{\quad \longrightarrow \quad \frac{3x+1}{2+4x}}
\end{align*}
\end{frame}

\begin{frame}{Example 1}
(b) \quad $\dfrac{\left(1-\dfrac{5}{x^2}\right)}{\left(\dfrac{2}{x^2}-7\right)} $
\onslide<2->{\quad LCTD is $x^2$}
\begin{align*}
\onslide<3->{\dfrac{\left(1-\dfrac{5}{x^2}\right)}{\left(\dfrac{2}{x^2}-7\right)} &\left(\dfrac{x^2}{x^2}\right)}
\onslide<4->{\quad \longrightarrow \quad \frac{x^2-5}{2-7x^2}}
\end{align*}
\end{frame}

\begin{frame}{Example 1}
(c)	\quad	$\dfrac{\left(\dfrac{1}{x}+\dfrac{y}{x^2}\right)}{\left(\dfrac{1}{y}+\dfrac{x}{y^2}\right)}$
\onslide<2->{\quad LCTD is $x^2y^2$}
\begin{align*}
\onslide<3->{\dfrac{\left(\dfrac{1}{x}+\dfrac{y}{x^2}\right)}{\left(\dfrac{1}{y}+\dfrac{x}{y^2}\right)} \left(\dfrac{x^2y^2}{x^2y^2}\right)}
\onslide<4->{\quad \longrightarrow \quad \dfrac{xy^2+y^3}{x^2y+x^3}}
\end{align*} 
\end{frame}

\begin{frame}{Example 1c}
\begin{align*}
&\dfrac{xy^2+y^3}{x^2y+x^3}	\\[12pt]
\onslide<2->{&=\dfrac{y^2(x+y)}{x^2(y+x)}} \\[12pt]
\onslide<3->{&=\dfrac{y^2}{x^2}}
\end{align*}
\end{frame}

\begin{frame}{Example 1}
(d)	\quad $\dfrac{\left(\dfrac{x}{y}-1\right)}{\left(\dfrac{x^2}{y^2}-1\right)}$
\onslide<2->{LCTD is $y^2$}
\begin{align*}
\onslide<2->{\dfrac{\left(\dfrac{x}{y}-1\right)}{\left(\dfrac{x^2}{y^2}-1\right)}\left(\dfrac{y^2}{y^2}\right)}
\onslide<3->{\quad \longrightarrow \quad \dfrac{xy-y^2}{x^2-y^2}}
\end{align*}
\end{frame}

\begin{frame}{Example 1d}
\begin{align*}
&\dfrac{xy-y^2}{x^2-y^2}	\\[12pt]
\onslide<2->{&= \dfrac{y(x-y)}{(x+y)(x-y)}}	\\[12pt]
\onslide<3->{&= \dfrac{y}{x+y}}
\end{align*}
\end{frame}

\begin{frame}{Example 1}
(e)	\quad $\dfrac{\left(\dfrac{1}{x+7}-\dfrac{1}{x}\right)}{7}$
\onslide<2->{\quad LCTD is $x(x+7)$}	\newline\\
\begin{align*}
\onslide<3->{&\dfrac{\left(\dfrac{1}{x+7}-\dfrac{1}{x}\right)}{7} \left(\dfrac{x(x+7)}{x(x+7)}\right)} \\[10pt]
\onslide<4->{&= \dfrac{x-(x+7)}{7x(x+7)}}		\\[10pt]
\onslide<5->{&= \dfrac{x-x-7}{7x(x+7)}}
\end{align*}
\end{frame}

\begin{frame}{Example 1e}
\begin{align*}
&\dfrac{x-x-7}{7x(x+7)}		\\[12pt]
\onslide<2->{&= \dfrac{-7}{7x(x+7)}}	\\[12pt]
\onslide<3->{&= \dfrac{-1}{x(x+7)}}
\end{align*}
\end{frame}

\begin{frame}{Example 1}
(f) \quad $\dfrac{\left(\dfrac{x+1}{x} + \dfrac{x+1}{x-1} \right)}{\left( \dfrac{x+2}{x} - \dfrac{2}{x-1}\right)}$
\onslide<2->{\quad LCTD is $x(x-1)$}	\newline\\
\begin{align*}
\onslide<3->{
\dfrac{\left(\dfrac{x+1}{x} + \dfrac{x+1}{x-1} \right)}{\left( \dfrac{x+2}{x} - \dfrac{2}{x-1}\right)} \left(\dfrac{x(x-1)}{x(x-1)} \right)
}	\\[12pt]
\onslide<4->{
\dfrac{(x+1)(x-1) + x(x+1)}{(x+2)(x-1)-2x}
}
\end{align*}
\end{frame}

\begin{frame}{Example 1f}
\begin{align*}
& \dfrac{(x+1)(x-1) + x(x+1)}{(x+2)(x-1)-2x} \\[14pt]
\onslide<2->{
&=\dfrac{\onslide<3->{x^2-1}\onslide<4->{+x^2+x}}{\onslide<5->{x^2+x-2}\onslide<6->{-2x}}
}	\\[12pt]
\onslide<7->{&=
\dfrac{2x^2+x-1}{x^2-x-2}
}	\\[12pt]
\onslide<8->{&=
\dfrac{(2x-1)(x+1)}{(x-2)(x+1)}
}
\onslide<9->{\quad \longrightarrow \quad = \dfrac{2x-1}{x-2}}
\end{align*}
\end{frame}


\end{document}