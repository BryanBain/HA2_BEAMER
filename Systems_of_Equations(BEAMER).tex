\documentclass[t]{beamer}
\usetheme{Copenhagen}
\setbeamertemplate{headline}{} % remove toc from headers
\beamertemplatenavigationsymbolsempty

\usepackage{amsmath, bm, tcolorbox}

\title{Systems of Equations}
\author{}
\date{}

\AtBeginSection[]
{
  \begin{frame}
    \frametitle{Objectives}
    \tableofcontents[currentsection]
  \end{frame}
}

\begin{document}

\begin{frame} 
\maketitle
\end{frame}

\section{Write a system of equations using matrices}

\begin{frame}{Matrix Multiplication}
Previously, we looked at the method for multiplying matrices:
\[
\begin{bmatrix}
7	&	2	&	-1	\\
0	&	5	&	4	\\
-3	&	6	&	2	\\
\end{bmatrix}
\cdot
\begin{bmatrix}
2	\\ 1 \\ 3 \\
\end{bmatrix}
\onslide<2->{= \begin{bmatrix}
13 \\ 17 \\ 6 \\
\end{bmatrix}}
\] 
\onslide<3->{In this section, we will use matrix multiplication to solve a system of equations.}
\end{frame}

\begin{frame}
\begin{tcolorbox}[colframe=green!20!black, colback = green!30!white,title=\textbf{System of Equations}]
A \textbf{system of equations} consists of 2 or more equations with 2 or more variables.
\end{tcolorbox}
\vspace{6pt}
\onslide<2->{\textsc{Examples}}
\onslide<3->{
\begin{align*}
2x + 5y &= 8 \\
-7x - 3y &= 9 \\
\end{align*}
\begin{align*}
4x-9y+2z &= 10 \\
-x + z &= 15	\\
3x+10y-12z &= 0 \\
\end{align*}
}
\end{frame}

\begin{frame}{Writing a System of Equations Using Matrices}
We can use matrix multiplication with the coefficients and variables to write a system of equations using matrices.		\newline\\		\pause

So
\begin{align*}
2x + 5y &= 8 \\
-7x - 3y &= 9 \\
\end{align*}
\onslide<3->{
becomes
\[
\begin{bmatrix}
2	&	5	\\
-7	&	-3	\\
\end{bmatrix}
\begin{bmatrix}
x	\\
y	\\
\end{bmatrix}
=
\begin{bmatrix}
8	\\
9	\\
\end{bmatrix}
\]}
\end{frame}

\begin{frame}{Writing a System of Equations Using Matrices}
And 
\begin{align*}
4x-9y+2z &= 10 \\
-x + z &= 15	\\
3x+10y-12z &= 0 \\
\end{align*}
\onslide<2->{
becomes
\[
\begin{bmatrix}
4	&	-9	&	2	\\
-1	&	\alert{0}	&	1	\\
3	&	10	&	-12	\\
\end{bmatrix}
\begin{bmatrix}
x	\\
y	\\
z	\\
\end{bmatrix}
=
\begin{bmatrix}
10	\\
15	\\
0		\\
\end{bmatrix}
\]
}
\end{frame}

\begin{frame}{Example 1}
Write each of the following systems of equations using matrices.	\newline\\
(a)	\quad
\begin{align*}
4y	+ z &= -1 \\
-4x - 5y + 5z &= 0 \\
6x + 5y &= 25 \\
\end{align*}
\onslide<2->{
\[
\begin{bmatrix}
0	&	4	&	1	\\
-4	&	-5	&	5	\\
6	&	5	&	0	\\
\end{bmatrix}
\onslide<3->{
\begin{bmatrix}
x \\ y \\ z \\
\end{bmatrix}
}
\onslide<4->{
= \begin{bmatrix}
-1 \\ 0 \\ 25 \\
\end{bmatrix}
}
\]
}
\end{frame}

\begin{frame}{Example 1}
(b) \quad 
\begin{align*}
x - y &= -9 \\
-3x - 4y &= -8 \\
\end{align*}
\onslide<2->{
\[
\begin{bmatrix}
1 & -1 \\
-3 & -4 \\
\end{bmatrix}
\onslide<3->{
\begin{bmatrix}
x \\ y \\
\end{bmatrix}
}
\onslide<4->{
= \begin{bmatrix}
-9 \\ -8 \\
\end{bmatrix}
}
\]
}
\end{frame}

\section{Solve a system of equations using inverse matrices}

\begin{frame}{Solving Equations Using Inverse Operations}
Long ago, you learned that to solve something like
\[ 5x = 10 \]
you divide (the inverse operation of multiplication) both sides by 5.	\newline\\	\pause

We don't ``divide" matrices in this sense, but we do need to use an inverse operation to solve the previous examples.
\end{frame}

\begin{frame}{Identity Matrices}
\begin{tcolorbox}[colframe=green!20!black, colback = green!30!white,title=\textbf{Identity Matrix}]
An \textbf{identity matrix} is a square matrix with 1s along the diagonal from top left to bottom right, and 0s elsewhere.
\end{tcolorbox}
\vspace{10pt}	\pause
\[
I_2 = \begin{bmatrix}
1 & 0 \\
0 & 1 \\
\end{bmatrix}
\quad
I_3 = \begin{bmatrix}
1 & 0 & 0 \\
0 & 1 & 0 \\
0 & 0 & 1 \\
\end{bmatrix}
\quad
\text{etc.}
\]
\end{frame}

\begin{frame}{Inverse Matrix}
\begin{tcolorbox}[colframe=green!20!black, colback = green!30!white,title=\textbf{Inverse Matrix}]
The \textbf{inverse matrix} for square matrix $A$ is $A^{-1}$.
\end{tcolorbox}
\vspace{10pt}	\pause
Inverse matrices are such that 
\[
A \cdot A^{-1} = I
\]	\pause
and that if $AX = B$ where $A$ is the matrix of coefficients and $B$ are the constants on the right side, then
\[
X = A^{-1} \cdot B
\]
\pause
\emph{Note:} We will not discuss the techniques of how to actually find the inverse of a matrix without a calculator.
\end{frame}

\begin{frame}{Example 2}
Solve each of the following using matrices.	\newline\\
(a) \quad 
\begin{align*}
4y	+ z &= -1 \\
-4x - 5y + 5z &= 0 \\
6x + 5y &= 25 \\
\end{align*}
\onslide<2->{
\[
\begin{bmatrix}
0	&	4	&	1	\\
-4	&	-5	&	5	\\
6	&	5	&	0	\\
\end{bmatrix}
\begin{bmatrix}
x \\ y \\ z \\
\end{bmatrix}
= \begin{bmatrix}
-1 \\ 0 \\ 25 \\
\end{bmatrix}
\]
}
\end{frame}

\begin{frame}{Example 2}
\[
\begin{bmatrix}
x \\ y \\ z \\
\end{bmatrix}
= A^{-1}B
\]
\pause \vspace{6pt}
\[
\begin{bmatrix}
x \\ y \\ z \\
\end{bmatrix}
= \begin{bmatrix}
5 \\ -1 \\ 3 \\
\end{bmatrix}
\]
\pause \vspace{6pt}
\[x = 5, \quad y = -1, \quad z = 3\]
\end{frame}

\begin{frame}{Example 2}
(b) \quad 
\begin{align*}
x - y &= -9 \\
-3x - 4y &= -8 \\
\end{align*}
\onslide<2->{
\[
\begin{bmatrix}
1 & -1 \\
-3 & -4 \\
\end{bmatrix}
\begin{bmatrix}
x \\ y \\
\end{bmatrix}
= \begin{bmatrix}
-9 \\ -8 \\
\end{bmatrix}
\]
}
\end{frame}

\begin{frame}{Example 2}
\[
\begin{bmatrix}
x \\ y \\
\end{bmatrix}
= A^{-1}B
\]
\pause \vspace{6pt}
\[
\begin{bmatrix}
x \\ y
\end{bmatrix}
= \begin{bmatrix}
-4 \\ 5
\end{bmatrix}
\]
\pause \vspace{6pt}
\[x = -4, \quad y = 5\]
\end{frame}

\begin{frame}{Example 2}
(c) \quad 
\begin{align*}
x - 3y - 5z &= -18 \\
-2x + 3y + 5z &= 23 \\
-x + 3y + 6z &= 17 \\
\end{align*}
\onslide<2->{
\[
\begin{bmatrix}
1 & -3 & - 5 \\
-2 & 3 & 5 \\
-1 & 3 & 6 \\
\end{bmatrix}
\cdot
\begin{bmatrix}
x \\ y \\ z \\
\end{bmatrix}
= \begin{bmatrix}
-18 \\ 23 \\ 17 \\
\end{bmatrix}
\]
}
\end{frame}

\begin{frame}{Example 2}
\[
\begin{bmatrix}
x \\ y \\
\end{bmatrix}
= A^{-1}B
\]
\pause \vspace{6pt}
\[
\begin{bmatrix}
x \\ y \\ z
\end{bmatrix}
= \begin{bmatrix}
-5 \\ 6 \\ -1 \\
\end{bmatrix}
\]
\pause \vspace{6pt}
\[x = -5, \quad y = 6, \quad z = -1\]
\end{frame}


\section{Solve applications of systems of equations}

\begin{frame}{Applied Systems of Equations}
Setting up many applied systems of equations problems boils down to 
\begin{center}
unit rate $\times$ amount = total amount
\end{center}
For instance, \$2.00 per gallon times 5 gallons of gas costs a total amount of \$10.
\end{frame}

\begin{frame}{Example 3}
(a) \quad How many mL of a solution containing 10\% pure hydrochloric acid must be mixed with a solution containing 15\% hydrochloric acid to produce 30 mL of a solution that is 11\% hydrochloric acid?	\pause	\newline\\
$x$ = mL of the 10\% acid solution \\
$y$ = mL of the 15\% acid solution \pause
\begin{align*}
x + y &= 30 &\text{volume of total liquid} \\
0.10x + 0.15y &= 30(0.11) &\text{volume of total hydrocholic acid} \\
\end{align*}	
\end{frame}

\begin{frame}{Example 3}
\[
\begin{bmatrix}
1	&	1	\\
0.1	&	0.15	\\
\end{bmatrix}
\cdot 
\begin{bmatrix}
x \\ y 
\end{bmatrix}
= \begin{bmatrix}
30 \\ 3.3 \\
\end{bmatrix}
\]
\vspace{6pt}
\onslide<2->{
\[
\begin{bmatrix}
x \\ y \\
\end{bmatrix}
= \begin{bmatrix}
24 \\ 6 \\
\end{bmatrix}
\]
}
\newline\\ \pause
We need to mix 24 mL of a 10\% hydrochloric acid solution with 6 mL of a 15\% hydrochloric acid solution.
\end{frame}

\begin{frame}{Example 3}
(b) \quad A coffee shop wants to sell 8 pound bags of a coffee blend for \$28.64. They will do this by blending coffee that costs \$2.25 per pound with coffee that costs \$4.00 per pound. How much of each type of coffee should they use in the blend?	\newline\\	\pause
$x$ = number of pounds of cheaper coffee (\$2.25 per pound) \\
$y$ = number of pounds of more expensive coffee (\$4.00 per pound) \pause
\begin{align*}
x + y &= 8  &\text{weight of each bag} \\
2.25x + 4.00y &= 28.64 &\text{total cost} \\
\end{align*}
\end{frame}

\begin{frame}{Example 3}
\[
\begin{bmatrix}
1	&	1	\\
2.25 & 4	\\
\end{bmatrix}
\cdot 
\begin{bmatrix}
x \\ y 
\end{bmatrix}
= \begin{bmatrix}
8 \\ 28.64 \\
\end{bmatrix}
\]
\vspace{6pt}
\onslide<2->{
\[
\begin{bmatrix}
x \\ y \\
\end{bmatrix}
= \begin{bmatrix}
1.92 \\ 6.08 \\
\end{bmatrix}
\]
}
\newline\\ \pause
They will need 1.92 pounds of the cheaper coffee and 6.08 pounds of the more expensive coffee.
\end{frame}

\end{document}