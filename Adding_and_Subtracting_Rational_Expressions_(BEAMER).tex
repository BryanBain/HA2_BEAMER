\documentclass[t]{beamer}
\usetheme{Copenhagen}
\setbeamertemplate{headline}{} % remove toc from headers
\beamertemplatenavigationsymbolsempty

\usepackage{amsmath, bm, xcolor, tcolorbox}
\everymath{\displaystyle}

\title{Complex Fractions}
\author{}
\date{}

\AtBeginSection[]
{
  \begin{frame}
    \frametitle{Objectives}
    \tableofcontents[currentsection]
  \end{frame}
}

\begin{document}

\begin{frame} 
\maketitle
\end{frame}

\section{Add and subtract rational expressions with like denominators}

\begin{frame}{Fractions with like denominators}
Recall that to add or subtract fractions with {\color{blue}\textbf{like denominators}}, you keep the denominators and add (or subtract) the numerators. \newline\\	\pause

Remember to \alert{distribute the sign} to the numerator in the second fraction.
\end{frame}

\begin{frame}{Example 1}
Simplify each.	\newline\\
(a)	\quad	$\frac{x+6}{9x^3+54x^2} + \frac{x+2}{9x^3+54x^2}$
\begin{align*}
\onslide<2->{&\frac{x+6}{9x^3+54x^2} + \frac{x+2}{9x^3+54x^2}} \\[12pt]
\onslide<3->{&= \frac{x+6+(x+2)}{9x^3+54x^2}} \\[12pt]
\onslide<4->{&= \frac{2x+8}{9x^3+54x^2}}	\\[12pt]
\onslide<5->{&= \frac{2(x+4)}{9x^2(x+6)}}
\end{align*}
\end{frame}

\begin{frame}{Example 1}
(b) \quad $\frac{x-2}{2x^2-9x-18} + \frac{6x+1}{2x^2-9x-18}$
\begin{align*}
\onslide<2->{& \frac{x-2}{2x^2-9x-18} + \frac{6x+1}{2x^2-9x-18}}	\\[12pt]
\onslide<3->{&= \frac{x-2+(6x+1)}{2x^2-9x-18}} \\[12pt]
\onslide<4->{&= \frac{7x-1}{2x^2-9x-18}}	\\[12pt]
\onslide<5->{&= \frac{7x-1}{(2x-3)(x+6)}}
\end{align*}
\end{frame}

\begin{frame}{Example 1}
(c)	\quad	 $\frac{x+5}{6x+4}-\frac{4x-1}{6x+4}$
\begin{align*}
\onslide<2->{&\frac{x+5}{6x+4}-\frac{4x-1}{6x+4}}	\\[12pt]
\onslide<3->{&= \frac{x+5-(4x-1)}{6x+4}} \\[12pt]
\onslide<4->{&= \frac{x+5-4x+1}{6x+4}} \\[12pt]
\onslide<5->{&= \frac{-3x+6}{6x+4}} \\[12pt]
\onslide<6->{&= \frac{-3(x-2)}{2(3x+2)}}
\end{align*}
\end{frame}

\begin{frame}{Example 1}
(d)	\quad	$\frac{x-1}{3x^2-10x-8} - \frac{x+6}{3x^2-10x-8}$
\begin{align*}
\onslide<2->{& \frac{x-1}{3x^2-10x-8} - \frac{x+6}{3x^2-10x-8}}	\\[12pt]
\onslide<3->{&= \frac{x-1-(x+6)}{3x^2-10x-8}}	\\[12pt]
\onslide<4->{&= \frac{x-1-x-6}{3x^2-10x-8}}	\\[12pt]
\onslide<5->{&= \frac{-7}{3x^2-10x-8}}
\end{align*}
\end{frame}

\section{Add and subtract rational expressions with unlike denominators}

\begin{frame}{Adding and Subtracting Fractions with Unlike Denominators}
Recall that before adding or subtracting fractions with unlike denominators, {\color{blue}\textbf{you need to get a common denominator first.}}	\newline\\	\pause

For instance,
\begin{align*}
\onslide<2->{& \frac{2}{5} + \frac{1}{3}}	\\[12pt]
\onslide<3->{&= \frac{2(3)}{5(3)} + \frac{1(5)}{3(5)}} \\[12pt]
\onslide<4->{&= \frac{6}{15} + \frac{5}{15}} \\[12pt]
\onslide<5->{&= \frac{11}{15}}
\end{align*}
\end{frame}

\begin{frame}{Ancient Secret to This Method}
Notice you had to find the \alert{least common multiple} of the denominators 3 and 5 (which ended up being 15). \newline\\	\pause

For adding and subtracting rational expressions with unlike denominators, {\color{blue}\textbf{factor the denominators completely and multiply by factors that differ}}. (Easier to see that process in action than it is to understand it written down like that).
\end{frame}

\begin{frame}{Example 2}
Simplify each.	\newline\\
(a) \quad $\frac{6}{x+4	} + \frac{2}{3x+6}$
\begin{align*}
\onslide<2->{& \frac{6}{x+4} + \frac{2}{3x+6}} \\[12pt]
\onslide<3->{&= \frac{6}{x+4} + \frac{2}{3(x+2)}} \\[12pt]
\onslide<4->{&= \frac{6(\alert{3(x+2)})}{(x+4)(\alert{3(x+2)})} + \frac{2(\alert{x+4})}{3(x+2)(\alert{x+4})}}
\end{align*}
\end{frame}

\begin{frame}{Example 2a}
\[\frac{6(\alert{3(x+2)})}{(x+4)(\alert{3(x+2)})} + \frac{2(\alert{x+4})}{3(x+2)(\alert{x+4})}\]
\begin{align*}
\onslide<2->{& \frac{6(3x+6) + 2x+8}{3(x+2)(x+4)}}	\\[12pt]
\onslide<3->{&= \frac{18x+36+2x+8}{3(x+2)(x+4)}} \\[12pt]
\onslide<4->{&= \frac{20x+44}{3(x+2)(x+4)}}	\\[12pt]
\onslide<5->{&= \frac{4(5x+11)}{3(x+2)(x+4)}}
\end{align*}
\end{frame}

\begin{frame}{Example 2}
(b) \quad $\frac{3x}{2x-1} + \frac{7}{5x+3}$
\begin{align*}
\onslide<2->{& \frac{3x}{2x-1} + \frac{7}{5x+3}}	\\[12pt]
\onslide<3->{&= \frac{3x(\alert{5x+3})}{(2x-1)(\alert{5x+3})} + \frac{7(\alert{2x-1})}{(5x+3)(\alert{2x-1})}}		\\[12pt]
\onslide<4->{&= \frac{15x^2+9x+14x-7}{(2x-1)(5x+3)}}	 \\[12pt]
\onslide<5->{&= \frac{15x^2 + 23x - 7}{(2x-1)(5x+3)}}	\quad 
\onslide<6->{\longrightarrow\frac{15x^2+23x-7}{10x^2+x-3}}
\end{align*}
\end{frame}

\begin{frame}{Example 2}
(c) \quad $\frac{8x}{3x^2+8x-3} + \frac{9}{x^2-x-12}$
\begin{align*}
\onslide<2->{& \frac{8x}{(x+3)(3x-1)} + \frac{9}{(x+3)(x-4)}}		\\[12pt]
\onslide<3->{&= \frac{8x(\alert{x-4})}{(x+3)(3x-1)(\alert{x-4})} + \frac{9(\alert{3x-1})}{(x+3)(x-4)(\alert{3x-1})}}	\\[12pt]
\onslide<4->{&= \frac{8x^2-32x+27x-9}{(x+3)(3x-1)(x-4)}}	\\[12pt]
\onslide<5->{&= \frac{8x^2-5x-9}{(x+3)(3x-1)(x-4)}}
\end{align*}
\end{frame}

\begin{frame}{Example 2}
(d) \quad $\frac{x}{x+3} - \frac{5}{x-1}$
\begin{align*}
\onslide<2->{& \frac{x}{x+3} - \frac{5}{x-1}} \\[12pt]
\onslide<3->{&= \frac{x(\alert{x-1})}{(x+3)(\alert{x-1})} - \frac{5(\alert{x+3})}{(x-1)(\alert{x+3})}}		\\[12pt]
\onslide<4->{&= \frac{x^2-x-(5x+15)}{(x+3)(x-1)}}	\\[12pt]
\onslide<5->{&= \frac{x^2-6x-15}{(x+3)(x-1)}} 
\onslide<6->{\quad \longrightarrow \quad \frac{x^2-6x-15}{x^2+2x-3}}
\end{align*}
\end{frame}

\begin{frame}{Example 2}
(e) \quad $\frac{x}{2x^2+5x-12} - \frac{3}{2x^2-19x+24}$
\begin{align*}
\onslide<2->{& \frac{x}{2x^2+5x-12} - \frac{3}{2x^2-19x+24}} \\[12pt]
\onslide<3->{&= \frac{x}{(2x-3)(x+4)} - \frac{3}{(2x-3)(x-8)}}	\\[12pt]
\onslide<4->{&=  \frac{x(\alert{x-8})}{(2x-3)(x+4)(\alert{x-8})} - \frac{3(\alert{x+4})}{(2x-3)(x-8)(\alert{x+4})}}	\\[12pt]
\onslide<5->{&= \frac{x^2-8x-(3x+12)}{(2x-3)(x+4)(x-8)}}
\end{align*}
\end{frame}

\begin{frame}{Example 2e}
\[	\frac{x^2-8x-(3x+12)}{(2x-3)(x+4)(x-8)}\]
\begin{align*}
\onslide<2->{& \frac{x^2-11x-12}{(2x-3)(x+4)(x-8)}}		\\[12pt]
\onslide<3->{&= \frac{(x-12)(x+1)}{(2x-3)(x+4)(x-8)}}
\end{align*}
\end{frame}
\end{document}