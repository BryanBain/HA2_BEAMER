\documentclass[t]{beamer}
\usetheme{Copenhagen}
\setbeamertemplate{headline}{} % remove toc from headers
\beamertemplatenavigationsymbolsempty

\usepackage{amsmath, tikz, bm, pgfplots, tcolorbox}
\pgfplotsset{compat = 1.16}

\title{Quadratic Formula}
\author{}
\date{}

\AtBeginSection[]
{
  \begin{frame}
    \frametitle{Objectives}
    \tableofcontents[currentsection]
  \end{frame}
}

\begin{document}

\begin{frame} 
\maketitle
\end{frame}

\section{Solve quadratic equations using the quadratic formula}

\begin{frame}{The Quadratic Formula}
The \alert{quadratic formula} can be used to solve \underline{any} quadratic equation that is equal to 0.	\newline\\	\pause

\begin{center}
For $ax^2 + bx + c = 0$
\end{center}

\begin{tcolorbox}[colback=red!10!white, colframe=red!60!black, title=Quadratic Formula]
\[x = \frac{-b \pm \sqrt{b^2-4ac}}{2a}\]
\end{tcolorbox}
\end{frame}

\begin{frame}{Example 1}
Solve each of the following using the quadratic formula. Exact answers only.	\newline\\
(a) \quad $3x^2 + 8x - 28 = 0$		\pause
\[ a = 3 \quad b = 8 \quad c = -28 \]
\begin{align*}
\onslide<3->{x &= \frac{-8 \pm \sqrt{8^2-4(3)(-28)}}{2(3)}} \\[10pt]
\onslide<4->{x &= \frac{-8 \pm \sqrt{400}}{6}} \\[10pt]
\onslide<5->{x &= \frac{-8 \pm 20}{6}}
\end{align*}
\end{frame}

\begin{frame}{Example 1}
\[x = \frac{-8 \pm 20}{6} \]
\begin{align*}
\onslide<2->{x &= \frac{-8+20}{6} & x &= \frac{-8-20}{6}} \\[10pt]
\onslide<3->{x &= 2 & x &= -\frac{14}{3}}
\end{align*}
\end{frame}

\begin{frame}{Example 1}
(b) \quad $5x^2 + 9x - 5 = 0$ \pause
\[ a = 5 \quad b = 9 \quad c = -5 \]
\begin{align*}
\onslide<3->{x &= \frac{-9 \pm \sqrt{9^2-4(5)(--5)}}{2(5)}} \\[10pt]
\onslide<4->{x &= \frac{-9 \pm \sqrt{181}}{10}}
\end{align*}
\onslide<5->{\[x = \frac{-9 + \sqrt{181}}{10} \quad x = \frac{-9 - \sqrt{181}}{10}\]}
\end{frame}

\begin{frame}{Example 1}
(c) \quad $6x^2 - 6x - 15 = -10$ \pause
\[ 6x^2 - 6x - 5 = 0 \]	\pause
\[a = 6 \quad b = -6 \quad c = -5 \]
\begin{align*}
\onslide<4->{x &= \frac{6 \pm \sqrt{6^2-4(6)(-5)}}{2(6)}} \\[10pt]
\onslide<5->{x &= \frac{6 \pm \sqrt{156}}{12}}
\end{align*}
\end{frame}

\begin{frame}{Example 1}
\[x = \frac{6 \pm \sqrt{156}}{12} \]
\begin{align*}
\onslide<2->{x &= \frac{6 \pm 2\sqrt{39}}{12}} \\[10pt]
\onslide<3->{x &= \frac{2\left(3\pm \sqrt{39}\right)}{12}} \\[10pt]
\onslide<4->{x &= \frac{3 \pm \sqrt{39}}{6}}
\end{align*}
\end{frame}

\begin{frame}{Example 1}
(d) \quad $7x^2 + 20x - 8 = -3x^2 - 1 + 10x$	\pause
\[ 10x^2 + 10x - 7 \]	\pause
\[a = 10 \quad b = 10 \quad c = -7\]
\begin{align*}
\onslide<4->{x &= \frac{-10 \pm \sqrt{10^2-4(10)(-7)}}{2(10)}} \\[10pt]
\onslide<5->{x &= \frac{-10 \pm \sqrt{380}}{20}}
\end{align*}
\end{frame}

\begin{frame}{Example 1}
\[x = \frac{-10 \pm \sqrt{380}}{20} \]
\begin{align*}
\onslide<2->{x &= \frac{-10 \pm 2\sqrt{95}}{20}} \\[10pt]
\onslide<3->{x &= \frac{2\left(-5\pm \sqrt{95}\right)}{20}} \\[10pt]
\onslide<4->{x &= \frac{-5 \pm \sqrt{95}}{10}}
\end{align*}
\end{frame}

\section{Use the discriminant to determine the types of solutions to a quadratic equation}

\begin{frame}{The Discriminant}
The expression {\color{blue}$b^2-4ac$} in the square root is called the \alert{discriminant}. It can tell us about the solutions to a quadratic equation:	\newline\\	\pause

\begin{itemize}
	\item \textbf{Discriminant is negative:} No real values of $x$ make the original equation true. \newline\\ \pause
	\item \textbf{Discriminant is 0:} There is one value of $x$ (called a \emph{double root}). \newline\\ \pause
	\item \textbf{Discriminant is positive:} There are 2 unique answers for $x$.
\end{itemize}
\end{frame}

\begin{frame}{The Discriminant}
In addition, if $\sqrt{b^2 - 4ac}$ equals a \alert{rational number}, then the quadratic equation is factorable over the integers (only use integers in your factoring).
\end{frame}

\end{document}